\documentclass{common/cs157}
\usepackage{hyperref}
% \usepackage{clrscode}
\usepackage{tikz}
\usepackage{graphicx}
\usepackage{listings}


\usepackage{amsmath}
\usepackage{amsfonts}
\usepackage{amssymb}

\usepackage{algorithmicx}
\usepackage{algorithm}
\usepackage{algpseudocode}

\usepackage[shortlabels]{enumitem}



% comment this in if you want to compile the solution key:
% \sol


\hwk{2}
\due{September 28, 2021 - 14:30 ET}


\begin{document}

\homeworkhandin % this is in common/cs157.cls if you need to edit it

\begin{problem}{1}
3-Mergesort is a variation of Mergesort, such that:
\begin{itemize}
    \item In the divide phase: the input is split into 3 sub-sequences on which 3-Mergesort is recursively invoked;
    \item In the conquer -merge- phase: the three sorted lists are merged in a single sorted list, using a variation of the Merging algorithm called 3-Merge. 
\end{itemize}
For the sake of simplicity, assume that the size of the sequence of elements to be sorted is a perfect power of 3.
\begin{enumerate}
    \item Provide a detailed, but concise, presentation of 3-Mergesort.
    \item Consider a k-way mergesort. Provide a generalization of the number of comparisons executed in the worst case. Please count the \emph{actual} number of comparisons.  
    \item Is 3-Mergesort faster than standard Mergesort? Is it asymptotically faster? (\textbf{Hint:} you may want to think about the actual constant terms vs the asymptotic notation).
\end{enumerate}
\end{problem}

\newpage

\begin{problem}{2}
\begin{enumerate}
    \item Assume that we want to sort sequences such that each element is placed in the input at most $k$ locations-away from its placement in the ordered sequence. Design a modification of Selection sort which runs in time O(nk).
    \item Under the same assumption, what can we say about the worst-case runtime of Insertion Sort?
    \item For input sequences exhibiting this property, which among Mergesort, Insertion Sort and Standard Selection sort offered better performance? Motivate your answer. 
    \item Assume now that we are looking to sort input sequences in which there is a subsequence of size $k$ which is already sorted. Show that using one among Selection, Insertion, or Bubble sort it is possible to sort such inputs in O(n(n-k)) steps.
\end{enumerate}
\end{problem}

\newpage

\begin{problem}{3}
\begin{enumerate}
    \item Say we are given a sorted list S with n items, i.e. n entries of the form (key, value) whose keys are strictly increasing with the indexing. Consider a “scrambled” version of it $S’$ in which keys may have been moved by at most two locations: that is, keys previously in index $i$ may be indexed within the range $[i-2, i+2]$ in $S’$. Provide a searching algorithm that still provides the functionality of the searching algorithm discussed in class with runtime O(log n). It should output the value associated with the key that is being searched for. Provide proof for the correctness and the runtime.
\end{enumerate}
\end{problem}

\end{document}